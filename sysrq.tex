\documentclass[article,letterpaper]{memoir}
\usepackage[final]{microtype}
\usepackage{hyperref}
\usepackage{longtable}
\setlrmargins{*}{*}{1}
\setulmarginsandblock{1.5in}{*}{1}
\checkandfixthelayout
\let\subsection\section
\let\section\chapter
\begin{document}
\section{Linux Magic System Request Key
Hacks}\label{linux-magic-system-request-key-hacks}

Documentation for sysrq.c

\iffalse
\subsection{What is the magic SysRq
key?}\label{what-is-the-magic-sysrq-key}

It is a `magical' key combo you can hit which the kernel will respond to
regardless of whatever else it is doing, unless it is completely locked
up.

\subsection{How do I enable the magic SysRq
key?}\label{how-do-i-enable-the-magic-sysrq-key}

You need to say ``yes'' to `Magic SysRq key (CONFIG\_MAGIC\_SYSRQ)' when
configuring the kernel. When running a kernel with SysRq compiled in,
/proc/sys/kernel/sysrq controls the functions allowed to be invoked via
the SysRq key. The default value in this file is set by the
CONFIG\_MAGIC\_SYSRQ\_DEFAULT\_ENABLE config symbol, which itself
defaults to 1. Here is the list of possible values in
/proc/sys/kernel/sysrq:

\begin{quote}
\begin{itemize}
\item
  0 - disable sysrq completely
\item
  1 - enable all functions of sysrq
\item
  \textgreater{}1 - bitmask of allowed sysrq functions (see below for
  detailed function description):

\begin{verbatim}
2 =   0x2 - enable control of console logging level
4 =   0x4 - enable control of keyboard (SAK, unraw)
8 =   0x8 - enable debugging dumps of processes etc.
\end{verbatim}

  \begin{quote}
  16 = 0x10 - enable sync command 32 = 0x20 - enable remount read-only
  64 = 0x40 - enable signalling of processes (term, kill, oom-kill)
  \end{quote}

  \begin{quote}
  128 = 0x80 - allow reboot/poweroff 256 = 0x100 - allow nicing of all
  RT tasks
  \end{quote}
\end{itemize}
\end{quote}

You can set the value in the file by the following command:

\begin{verbatim}
echo "number" >/proc/sys/kernel/sysrq
\end{verbatim}

The number may be written here either as decimal or as hexadecimal with
the 0x prefix. CONFIG\_MAGIC\_SYSRQ\_DEFAULT\_ENABLE must always be
written in hexadecimal.

Note that the value of \texttt{/proc/sys/kernel/sysrq} influences only
the invocation via a keyboard. Invocation of any operation via
\texttt{/proc/sysrq-trigger} is always allowed (by a user with admin
privileges).

\subsection{How do I use the magic SysRq
key?}\label{how-do-i-use-the-magic-sysrq-key}

On x86 - You press the key combo
ALT-SysRq-\textless{}command key\textgreater{}.

\begin{description}
\item[Some]
keyboards may not have a key labeled `SysRq'. The `SysRq' key is also
known as the `Print Screen' key. Also some keyboards cannot
\item[handle so many keys being pressed at the same time, so you might]
have better luck with press Alt, press SysRq, release SysRq, press
\textless{}command key\textgreater{}, release everything.
\end{description}

On SPARC - You press ALT-STOP-\textless{}command key\textgreater{}, I
believe.

\begin{description}
\item[On the serial console (PC style standard serial ports only)]
You send a \texttt{BREAK}, then within 5 seconds a command key. Sending
\texttt{BREAK} twice is interpreted as a normal BREAK.
\item[On PowerPC]
\begin{description}
\item[Press ALT - Print Screen (or F13) -
\textless{}command key\textgreater{},]
Print Screen (or F13) - \textless{}command key\textgreater{} may
suffice.
\end{description}
\item[On other]
\begin{description}
\item[If you know of the key combos for other architectures, please]
let me know so I can add them to this section.
\end{description}
\item[On all]
write a character to /proc/sysrq-trigger. e.g.:

\begin{verbatim}
echo t > /proc/sysrq-trigger
\end{verbatim}
\end{description}
\fi

\subsection{\texorpdfstring{What are the `command'
keys?}{What are the command keys?}}\label{what-are-the-command-keys}

\begin{longtable}[]{@{}ll@{}}
\toprule
Command & Function\tabularnewline
\midrule
\endhead
\texttt{b} & Will immediately reboot the system without syncing or
unmounting\tabularnewline
& your disks.\tabularnewline
\texttt{c} & Will perform a system crash by a NULL pointer
dereference.\tabularnewline
& A crashdump will be taken if configured.\tabularnewline
\texttt{d} & Shows all locks that are held.\tabularnewline
\texttt{e} & Send a SIGTERM to all processes, except for
init.\tabularnewline
\texttt{f} & Will call the oom killer to kill a memory hog process, but
do not\tabularnewline
pani & c if nothing can be killed.\tabularnewline
\texttt{g} & Used by kgdb (kernel debugger)\tabularnewline
\texttt{h} & Will display help (actually any other key than those
listed\tabularnewline
& here will display help. but \texttt{h} is easy to remember
:-)\tabularnewline
\texttt{i} & Send a SIGKILL to all processes, except for
init.\tabularnewline
\texttt{j} & Forcibly ``Just thaw it'' - filesystems frozen by the
FIFREEZE ioctl.\tabularnewline
\texttt{k} & Secure Access Key (SAK) Kills all programs on the current
virtual\tabularnewline
& console. NOTE: See important comments below in SAK
section.\tabularnewline
\texttt{l} & Shows a stack backtrace for all active CPUs.\tabularnewline
\texttt{m} & Will dump current memory info to your
console.\tabularnewline
\texttt{n} & Used to make RT tasks nice-able\tabularnewline
\texttt{o} & Will shut your system off (if configured and
supported).\tabularnewline
\texttt{p} & Will dump the current registers and flags to your
console.\tabularnewline
\texttt{q} & Will dump per CPU lists of all armed hrtimers (but NOT
regular\tabularnewline
& timer\_list timers) and detailed information about all\tabularnewline
& clockevent devices.\tabularnewline
\texttt{r} & Turns off keyboard raw mode and sets it to
XLATE.\tabularnewline
\texttt{s} & Will attempt to sync all mounted
filesystems.\tabularnewline
\texttt{t} & Will dump a list of current tasks and their information to
your\tabularnewline
& console.\tabularnewline
\texttt{u} & Will attempt to remount all mounted filesystems
read-only.\tabularnewline
\texttt{v} & Forcefully restores framebuffer console\tabularnewline
\texttt{v} & Causes ETM buffer dump {[}ARM-specific{]}\tabularnewline
\texttt{w} & Dumps tasks that are in uninterruptable (blocked)
state.\tabularnewline
\texttt{x} & Used by xmon interface on ppc/powerpc
platforms.\tabularnewline
& Show global PMU Registers on sparc64.\tabularnewline
& Dump all TLB entries on MIPS.\tabularnewline
\texttt{y} & Show global CPU Registers {[}SPARC-64
specific{]}\tabularnewline
\texttt{z} & Dump the ftrace buffer\tabularnewline
\texttt{0}-\texttt{9} & Sets the console log level, controlling which
kernel messages\tabularnewline
& will be printed to your console. (\texttt{0}, for example would
make\tabularnewline
& it so that only emergency messages like PANICs or OOPSes
would\tabularnewline
& make it to your console.)\tabularnewline
\bottomrule
\end{longtable}

\clearpage
\subsection{Okay, so what can I use them
for?}\label{okay-so-what-can-i-use-them-for}

Well, unraw(r) is very handy when your X server or a svgalib program
crashes.

sak(k) (Secure Access Key) is useful when you want to be sure there is
no trojan program running at console which could grab your password when
you would try to login. It will kill all programs on given console, thus
letting you make sure that the login prompt you see is actually the one
from init, not some trojan program.

In its true form it is not a true SAK like the one in a c2 compliant
system, and it should not be mistaken as such.

It seems others find it useful as (System Attention Key) which is useful
when you want to exit a program that will not let you switch consoles.
(For example, X or a svgalib program.)

\texttt{reboot(b)} is good when you're unable to shut down. But you
should also \texttt{sync(s)} and \texttt{umount(u)} first.

\texttt{crash(c)} can be used to manually trigger a crashdump when the
system is hung. Note that this just triggers a crash if there is no dump
mechanism available.

\texttt{sync(s)} is great when your system is locked up, it allows you
to sync your disks and will certainly lessen the chance of data loss and
fscking. Note that the sync hasn't taken place until you see the ``OK''
and ``Done'' appear on the screen. (If the kernel is really in strife,
you may not ever get the OK or Done message\ldots{})

\texttt{umount(u)} is basically useful in the same ways as
\texttt{sync(s)}. I generally \texttt{sync(s)}, \texttt{umount(u)}, then
\texttt{reboot(b)} when my system locks. It's saved me many a fsck.
Again, the unmount (remount read-only) hasn't taken place until you see
the ``OK'' and ``Done'' message appear on the screen.

The loglevels \texttt{0}-\texttt{9} are useful when your console is
being flooded with kernel messages you do not want to see. Selecting
\texttt{0} will prevent all but the most urgent kernel messages from
reaching your console. (They will still be logged if syslogd/klogd are
alive, though.)

\texttt{term(e)} and \texttt{kill(i)} are useful if you have some sort
of runaway process you are unable to kill any other way, especially if
it's spawning other processes.

``just thaw \texttt{it(j)}'' is useful if your system becomes
unresponsive due to a frozen (probably root) filesystem via the FIFREEZE
ioctl.

\iffalse
\subsection{\texorpdfstring{Sometimes SysRq seems to get `stuck' after
using it, what can I
do?}{Sometimes SysRq seems to get stuck after using it, what can I do?}}\label{sometimes-sysrq-seems-to-get-stuck-after-using-it-what-can-i-do}

That happens to me, also. I've found that tapping shift, alt, and
control on both sides of the keyboard, and hitting an invalid sysrq
sequence again will fix the problem. (i.e., something like alt-sysrq-z).
Switching to another virtual console (ALT+Fn) and then back again should
also help.

\subsection{I hit SysRq, but nothing seems to happen, what's
wrong?}\label{i-hit-sysrq-but-nothing-seems-to-happen-whats-wrong}

There are some keyboards that produce a different keycode for SysRq than
the pre-defined value of 99 (see \texttt{KEY\_SYSRQ} in
\texttt{include/linux/input.h}), or which don't have a SysRq key at all.
In these cases, run \texttt{showkey\ -s} to find an appropriate scancode
sequence, and use
\texttt{setkeycodes\ \textless{}sequence\textgreater{}\ 99} to map this
sequence to the usual SysRq code (e.g., \texttt{setkeycodes\ e05b\ 99}).
It's probably best to put this command in a boot script. Oh, and by the
way, you exit \texttt{showkey} by not typing anything for ten seconds.

\subsection{I want to add SysRQ key events to a module, how does it
work?}\label{i-want-to-add-sysrq-key-events-to-a-module-how-does-it-work}

In order to register a basic function with the table, you must first
include the header \texttt{include/linux/sysrq.h}, this will define
everything else you need. Next, you must create a
\texttt{sysrq\_key\_op} struct, and populate it with A) the key handler
function you will use, B) a help\_msg string, that will print when SysRQ
prints help, and C) an action\_msg string, that will print right before
your handler is called. Your handler must conform to the prototype in
`sysrq.h'.

After the \texttt{sysrq\_key\_op} is created, you can call the kernel
function
\texttt{register\_sysrq\_key(int\ key,\ struct\ sysrq\_key\_op\ *op\_p);}
this will register the operation pointed to by \texttt{op\_p} at table
key `key', if that slot in the table is blank. At module unload time,
you must call the function
\texttt{unregister\_sysrq\_key(int\ key,\ struct\ sysrq\_key\_op\ *op\_p)},
which will remove the key op pointed to by `op\_p' from the key `key',
if and only if it is currently registered in that slot. This is in case
the slot has been overwritten since you registered it.

The Magic SysRQ system works by registering key operations against a key
op lookup table, which is defined in `drivers/tty/sysrq.c'. This key
table has a number of operations registered into it at compile time, but
is mutable, and 2 functions are exported for interface to it:

\begin{verbatim}
register_sysrq_key and unregister_sysrq_key.
\end{verbatim}

Of course, never ever leave an invalid pointer in the table. I.e., when
your module that called register\_sysrq\_key() exits, it must call
unregister\_sysrq\_key() to clean up the sysrq key table entry that it
used. Null pointers in the table are always safe. :)

If for some reason you feel the need to call the handle\_sysrq function
from within a function called by handle\_sysrq, you must be aware that
you are in a lock (you are also in an interrupt handler, which means
don't sleep!), so you must call \texttt{\_\_handle\_sysrq\_nolock}
instead.

\subsection{When I hit a SysRq key combination only the header appears
on the
console?}\label{when-i-hit-a-sysrq-key-combination-only-the-header-appears-on-the-console}

Sysrq output is subject to the same console loglevel control as all
other console output. This means that if the kernel was booted `quiet'
as is common on distro kernels the output may not appear on the actual
console, even though it will appear in the dmesg buffer, and be
accessible via the dmesg command and to the consumers of
\texttt{/proc/kmsg}. As a specific exception the header line from the
sysrq command is passed to all console consumers as if the current
loglevel was maximum. If only the header is emitted it is almost certain
that the kernel loglevel is too low. Should you require the output on
the console channel then you will need to temporarily up the console
loglevel using alt-sysrq-8 or:

\begin{verbatim}
echo 8 > /proc/sysrq-trigger
\end{verbatim}

Remember to return the loglevel to normal after triggering the sysrq
command you are interested in.

\subsection{I have more questions, who can I
ask?}\label{i-have-more-questions-who-can-i-ask}

\begin{description}
\item[Just ask them on the linux-kernel mailing list:]
\href{mailto:linux-kernel@vger.kernel.org}{\nolinkurl{linux-kernel@vger.kernel.org}}
\end{description}
\fi

\subsection{Credits}\label{credits}

Written by Mydraal
\textless{}\href{mailto:vulpyne@vulpyne.net}{\nolinkurl{vulpyne@vulpyne.net}}\textgreater{}
Updated by Adam Sulmicki
\textless{}\href{mailto:adam@cfar.umd.edu}{\nolinkurl{adam@cfar.umd.edu}}\textgreater{}
Updated by Jeremy M. Dolan
\textless{}\href{mailto:jmd@turbogeek.org}{\nolinkurl{jmd@turbogeek.org}}\textgreater{}
2001/01/28 10:15:59 Added to by Crutcher Dunnavant
\textless{}\href{mailto:crutcher+kernel@datastacks.com}{\nolinkurl{crutcher+kernel@datastacks.com}}\textgreater{}
\end{document}
